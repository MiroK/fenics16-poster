 \documentclass[tikz,border=0cm]{standalone}
\usepackage{type1cm}
\usepackage{fp}
\usetikzlibrary{decorations.pathmorphing}
\usetikzlibrary{decorations.fractals}
\usetikzlibrary{calc}
\usetikzlibrary{shadows}

\usepackage{fetamont}
%\usepackage[rm,light]{roboto}
%\usepackage[scaled]{helvet}
%\usepackage{avant}
%\renewcommand{\familydefault}{\sfdefault}
\usepackage{cmbright}

\usepackage{bm,amsmath,amssymb}
\usepackage{soul}
\usepackage{minted}
\usepackage{tcolorbox}
\usepackage{wrapfig}
\usepackage{multirow}

\newcommand{\clm}{Cl\'{e}ment }
\newcommand{\norm}[1]{\lVert#1\rVert}

\edef\myfontscale{1.7}
\definecolor{simula}{RGB}{245,130,32}
\definecolor{amaranth}{rgb}{0.9, 0.17, 0.31}
\definecolor{amber}{rgb}{1.0, 0.75, 0.0}

\definecolor{edblue}{RGB}{4, 80, 124}
\definecolor{eyellow}{RGB}{248, 176, 23}
\definecolor{elblue}{RGB}{55, 172, 195}

\definecolor{bg}{rgb}{0.95,0.95,0.95}

\input{rescalefonts}

\usepackage{booktabs,dcolumn}
\usepackage{array}
\newcolumntype{T}{>{\begingroup\bfseries}r<{\endgroup}}
\newcolumntype{N}{D{.}{.}{-1}}
\newcolumntype{R}{D{.}{.}{2.4}}
\newcolumntype{K}{D{.}{.}{3.4}}
\newcommand{\imgcase}[2]{\includegraphics[width=0.5\linewidth]{#1_#2_base}%
\includegraphics[width=0.5\linewidth]{#1_#2_side}}
\usepackage[tableposition=top,font=footnotesize,labelfont=bf,sf]{caption}

\usepackage[margin=1in, paperwidth=48in, paperheight=48in]{geometry}

\begin{document}

\begin{tikzpicture}[x=70cm, y=100cm]
\fill[white, use as bounding box] (0, 0) rectangle (1, 1);

%%%%%%%%%%%%%%%%%%%%%%%%%%
% Title
%%%%%%%%%%%%%%%%%%%%%%%%%%
\begingroup
\node[inner sep=1cm, 
      fill=eyellow,
      anchor=north, 
      draw=elblue, line width=3.5mm]
      (title)
      at ($(current bounding box.north)$)
  {\begin{minipage}{0.88\textwidth}
   \begin{center}
   {\ffmfamily\bfseries\color{black}
   {\LARGE fenicstools} \\[0.4ex]
   {\LARGE smart add-ons for FEniCS}}

   \bigskip
   {\color{edblue}
     \textbf{\underline{M.~Mortensen}, M.~Kuchta}

   \vskip-5mm\footnotesize
   \texttt{mikaem@math.uio.no}}

   \end{center}
  \end{minipage}};

  \node[anchor=west] (foo) at ($(current bounding box.north west)+(7cm, -4.6cm)$)
  {
    \includegraphics[height=7cm]{graphics/fenicslogo}
  };
  \node[anchor=east] (foo) at ($(current bounding box.north east)+(-7cm, -4.6cm)$)
  {
    \includegraphics[height=7cm]{graphics/qrfenicstools}
  };
\endgroup

%%%%%%%%%%%%%%%%%%%%%%%%%%%%%%%%%
% Introduction, no frame
%%%%%%%%%%%%%%%%%%%%%%%%%%%%%%%%%
\node[anchor=north]
(motif title)
at ($(current bounding box.north)+(0cm,-10cm)$)
{
\begin{minipage}{0.6\textwidth}
\centering
{
  fenicstools is a collection of extensions to FEniCS library with particular
  focus on efficient visualization and postprocessing of results of large
  scale
  computations.
}
\end{minipage}
};

%%%%%%%%%%%%%%%%%%%%%%%%%%%%%%%%%
% Clement
%%%%%%%%%%%%%%%%%%%%%%%%%%%%%%%%%
\node[draw=none, anchor=north west]
(motif text)
at ( $(motif title.south west)+(1cm,0)$ )
{
};
%
\node[anchor=north west]
(clm title)
at ($(motif text.south west)-(1cm,0cm)$){
};
%
\node[draw=elblue, 
      fill=eyellow!20!white,
      line width=1mm, anchor=north west,
      rounded corners=4mm, inner sep=1cm,
      minimum height=20cm]
(clm)
at ( $(clm title.south west)+(1cm,0)$ )
{\begin{minipage}{40cm}   % 70 cm take 40
\vspace{1.0cm}
\footnotesize
  Evaluating quantities derived from primary unknowns, e.g. strain from
  displacement, is a frequent part of computational loops. Such quantities often
  lack the $H^2$ regularity required for nodal interpolation and therefore can
  be computed in FEniCS only by $L^2$ projection. An alternative method is the 
  \clm interpolation - a numerical technique for constructing interpolants of
  $H^1$ functions based on local regularization.
%

\vspace{1cm}
\begin{minipage}{0.58\textwidth}
\textbf{$\triangleright$ Figure}:
{fenicstools implements the lowest order \clm interpolation operator resulting
  in a CG$_1$ approximation of interpolated function $f$. The degree of freedom 
  at $x_j$ is computed as $v$ minimizing $\norm{f-v}^2_{0, {w}_j}$ over
  constant fields on patch $w_j$. The interpolation error is controlled on 
  the union $\tilde{w}_K$. Therefore no power $h$ is lost in the error estimates:
  \[
    \norm{u - I_h u}_{m, K} \leq C h^{1-m}_K\norm{u}_{1, \tilde{w}_K}
  \]
  for $u\in H^1$, $m=0, 1$ and $K$ and element of triangulation.
}
\end{minipage}
\hfill
\begin{minipage}{0.4\textwidth}
\begin{center}
\includegraphics[width=0.8\textwidth]{./graphics/domains}
\end{center}
\end{minipage}

\vspace{1cm}
\begin{minipage}{0.4\textwidth}
\begin{center}
\includegraphics[width=\textwidth]{./graphics/plot_Ih}\\
\includegraphics[width=\textwidth]{./graphics/plot_Ih_clement}
\end{center}
\end{minipage}
\hfill
\begin{minipage}{0.58\textwidth}
\textbf{$\triangleleft$ Figure}:{
  The local regularization/averaging procedure results in smearing of gradients. 
  However, the largest errors are localized near the boundaries where the
  interpolant fails to preserve the boundary values.
}\\
  \textbf{$\triangledown$ Figure}:{fenicstools supports \clm interpolation of 
  all\textsuperscript{*} valid UFL expressions. How about evaluating $w=\nabla\cdot(u\otimes\nabla v)$
  for $u\in\left[\text{CG}_1\right]^2$, $v\in\text{CG}_1$ or $w=\sin(\det\nabla
  u)$ where $u$ is a scalar field in $\mathbb{R}^3$?\\
  \vspace{0.5cm}
  \begin{center}
  \includegraphics[height=12cm]{./graphics/rate_plot}
  \end{center}
}
\end{minipage}

% % Implementation
%   Unlike $L^2$ projection, which requires solution of large linear system, \clm 
%   interpolant is constructed from local linear systems of size 1 assembled over
%   patches surrounding mesh vertices. In fenicstools this mapping is realized
%   more efficiently using a precomputed averaging operator $A$ such that 
%   $Ab(u)=I_h u$. Due to this choice the setup cost of the interpolant is higher
%   (4x) than that of $L^2$ projector. However, the subsequent (repetitive)
%   evaluation comes at a cost of a matrix-vector product.
% 
% \vspace{1cm}
% \begin{center}
% \begin{tabular}{l|ccccc}
% \hline
% \multirow{2}{*}{$n$} & \multicolumn{5}{c}{size}\\
% \cline{2-6}
%    &       132098 &   526338     &      2101250 &       8396802 &      14612418\\
% \hline
% %1  & (0.05, 0.28) & (0.23, 1.19) & (0.95, 5.04) & (4.20, 20.93) & (7.34, 39.79)\\
% %2  & (0.02, 0.18) & (0.08, 0.79) & (0.33, 3.08) & (1.11, 15.48) & (2.16, 27.95)\\
% %4  & (0.00, 0.12) & (0.02, 0.44) & (0.06, 1.73) & (0.28, 6.86)  & (0.80, 15.45)\\
% %8  & (0.00, 0.09) & (0.02, 0.40) & (0.02, 1.31) & (0.17, 4.78)  & (0.21, 11.36)\\
% %16 & (0.00, 0.06) & (0.00, 0.25) & (0.01, 0.88) & (0.04, 3.60)  & (0.09, 7.36) \\
% 1  & (0.07, 0.32) & (0.23, 1.21) & (0.99, 5.17) & (4.10, 20.93) & (7.46, 37.21)\\
% 2  & (0.03, 0.19) & (0.12, 0.76) & (0.66, 3.49) & (2.74, 14.04) & (3.95, 26.42)\\
% 4  & (0.04, 0.20) & (0.09, 0.45) & (0.68, 2.21) & (1.49, 10.80) & (2.03, 19.06)\\
% 8  & (0.02, 0.09) & (0.09, 0.43) & (0.33, 1.61) & (0.76, 8.46) & (1.82, 13.89)\\
% 16 & (0.02, 0.07) & (0.04, 0.32) & (0.18, 1.06) & (0.72, 7.46) & (1.65, 13.36)\\
% \hline
% \end{tabular}
% \end{center}

Unlike $L^2$ projection, which requires solution of large linear system, \clm 
interpolant is constructed from local linear systems of size 1 assembled over
patches surrounding mesh vertices.

\begin{wraptable}{r}{0.48\textwidth}
\begin{tabular}{l|ccc}
\hline
\multirow{2}{*}{$n$} & \multicolumn{3}{c}{size}\\
\cline{2-4}
   &       2101250 &       8396802 &      14612418\\
\hline
1  &  (0.99, 5.17) & (4.10, 20.93) & (7.46, 37.21)\\
2  &  (0.66, 3.49) & (2.74, 14.04) & (3.95, 26.42)\\
4  &  (0.68, 2.21) & (1.49, 10.80) & (2.03, 19.06)\\
8  &  (0.33, 1.61) & (0.76, 8.46) & (1.82, 13.89)\\
16 &  (0.18, 1.06) & (0.72, 7.46) & (1.65, 13.36)\\
\hline
\end{tabular}
\end{wraptable} 
%------------------------------------------
In fenicstools the mapping is realized more efficiently using a precomputed 
averaging operator $A$ such that $Ab(u)=I_h u$. Due to this choice the setup cost
of the interpolant is higher (4x) than that of $L^2$ projector. However, the 
subsequent (repetitive) evaluation comes at a cost of a matrix-vector product.
\textbf{$\triangleright$ Table}: CPU cost of action of $L^2$ projector vs. \clm
interpolator.
%
\footnotesize
\end{minipage}};
% 
\node[anchor=west, rounded corners=3mm, fill=eyellow]
(clm title)
at ($ (clm.north west)+(4cm, 0cm) $)
{
  \ffmfamily\textcolor{edblue}{\clm interpolation}
};
% 
% %%%% applications %%%%
% \node[anchor=north east]
% (applications title)
% at ($(motif text.south west)+(115.2cm,+0cm)$)
% {\bfseries\ffmfamily\Large\underline{Application examples}};
% 
% %%%% sensitivity analysis %%%%
% \node[draw=simula, line width=1mm, anchor=north west,
%       rounded corners=4mm, inner sep=1cm,
%       fill=simula!10!white, minimum height=25cm]
% (sensitivity)
% at ($(howitworks.north east)+(2cm,0)$)
% {\begin{minipage}{57cm}
% \begin{minipage}{0.6\textwidth}
% \footnotesize
% \vspace{1cm}
% Consider the time dependent heat equation
%     \begin{align*}
%          \frac{\partial u}{\partial t} - \nu \nabla^{2} u= 0
%              \quad & \textrm{in\phantom{r} } \Omega \times (0, T), \\
%             u = g  \quad & \textrm{for } \Omega \times \{0\}.
%     \end{align*}
% Here $\Omega$ is the Gray's Klein bottle, a closed 2D manifold embedded in 3D,
%     $T$ is the final time, $u$ is the
%     unknown temperature, $\nu$ is the thermal diffusivity, and $g$ is the initial
%     temperature.
% 
% The goal is to compute the sensitivity of the norm of temperature at the final time
%     \begin{equation*}
%         J(u) = \int_\Omega u(t=T)^2
%     \end{equation*}
% with respect to the initial temperature,  that is $\textrm{d} J / \textrm{d} g$.
% 
% \includegraphics[width=0.3\textwidth]{images/klein_bottle-init}
% \includegraphics[width=0.3\textwidth]{images/klein_bottle-final}
% \includegraphics[width=0.3\textwidth]{images/klein_bottle-sensitivity}
% \begin{minipage}{0.3\textwidth}
%     \centering
%     Initial temperature
% \end{minipage}
% \quad
% \begin{minipage}{0.3\textwidth}
%     \centering
%     Final temperature
% \end{minipage}
% \quad
% \begin{minipage}{0.3\textwidth}
%     \centering
%     Sensitivity
% \end{minipage}
% \end{minipage}
% \hfill
% \begin{minipage}{0.35\textwidth}
% \footnotesize
%     \begin{tcolorbox}
%         \begin{minted}{python}
% from dolfin import *
% from dolfin_adjoint import *
% 
% # Solve the forward system
% F = u*v*dx - u_old*v*dx +
%     dt*nu*inner(grad(v),grad(u))*dx
% while t <= T:
%     t += dt
%     solve(F == 0, u)
% 
% # Apply dolfin-adjoint
% m = Control(g)
% J = u**2*dx*dt[T]
% dJdm = compute_gradient(J, m)
% H = hessian(J, m)
%     \end{minted}
% \end{tcolorbox}
% \textbf{$\vartriangle$ \\Code}: Implementation excerpt (the code including the
% complete forward model has $37$ lines)
% 
% \end{minipage}
% 
% \end{minipage}};
% 
% \node[anchor=west, rounded corners=3mm, fill=simula]
% (sensitivity title)
% at ($ (sensitivity.north west)!3cm!(sensitivity.north east) $)
% {\ffmfamily Sensitivity analysis};
% 
% %%%% PDE-constrained optimization analysis %%%%
% \node[draw=simula, line width=1mm, anchor=north west,
%       rounded corners=4mm, inner sep=1cm,
%       fill=simula!10!white, minimum height=22cm]
% (opti)
% at ($(sensitivity.south west)-(0, 2cm)$)
% {\begin{minipage}{57cm}
%         \vspace{1cm}
% \begin{minipage}{0.5\textwidth}
%     \footnotesize
%      This topology optimization example minimizes the compliance
%     \begin{align*}
%         \int_{\Omega} fT + \alpha \int_{\Omega} \nabla a \cdot
%         \nabla a,
%     \end{align*}
%      subject to the Poisson equation with
%      mixed Dirichlet--Neumann
%      conditions
%     \begin{align*}
%            -\mathrm{div}(k(a) \nabla T) &= f  \qquad \mathrm{in} \ \Omega,           \\
%                              T &= 0  \qquad \mathrm{on} \ \partial \Omega_D,  \\
%                k(a) \nabla T &= 0  \qquad \mathrm{on} \ \partial \Omega_N,
%     \end{align*}
%      and additional control constraints
%     \begin{align*}
%               \int_{\Omega} a \le V \textrm{ and }
%               0 \le a(x) &\le 1  \qquad \forall x \in \Omega.
%     \end{align*}
% Here $\Omega$ is the unit square, $T$ is the temperature, $a$ is the control
% ($a(x) = 1$ means material, $a(x) = 0$ means no material), $f$ is a source term, $k(a)$ is the Solid Isotropic Material with Penalisation
% parameterization, $\alpha$ is a regularization term, and $V$ is the volume
% bound on the control.
% Physically, the problem is to find the material distribution $a$ that minimizes the integral of the temperature for a limited amount of conducting material.
% \end{minipage}
% \hfill
% \begin{minipage}{0.48\textwidth}
%     \begin{tcolorbox}
%     \footnotesize
% \begin{minted}{python}
% from dolfin import *
% from dolfin_adjoint import *
% # ...
% J = f*T*dx + alpha*inner(grad(a),grad(a))*dx
% m = Control(a)
% rf = ReducedFunctional(J, m)
% minimize(rf, method="SLSQP", bounds=...)
% \end{minted}
%     \end{tcolorbox}
% \vspace{5mm}
% \includegraphics[width=0.44\textwidth]{images/poisson-topology_2d}
% \hfill
% \begin{minipage}[b]{0.5\textwidth}
% \footnotesize
% \textbf{$\vartriangle$ \\Code}: Implementation excerpt (the full code uses
% the IPOPT optimization package and has $56$ lines) \\
% \textbf{$\triangleleft$ Figure}: Optimal material distribution $a$ for a unit square domain and $f=10^{-2}$
% \vspace{1.4cm}
% \end{minipage}
% \end{minipage}
% 
% 
% \end{minipage}};
% 
% \node[anchor=west, rounded corners=3mm, fill=simula]
% (opti title)
% at ($ (opti.north west)!3cm!(opti.north east) $)
% {\ffmfamily PDE-constrained optimization};
% 
% %%%% Generalized stability analysis %%%%
% \node[draw=simula, line width=1mm, anchor=north west,
%       rounded corners=4mm, inner sep=1cm,
%       fill=simula!10!white, minimum height=25cm]
% (stab)
% at ($(opti.south west)-(0, 2cm)$)
% {\begin{minipage}{57cm}
% \footnotesize
% \begin{minipage}{0.6\textwidth}
% \vspace{1cm}
%  This example performs a generalized stability analysis to find the
%  perturbations to an initial condition that grow the most over some finite
%  time. The governing equations are the two-dimensional
%  vorticity-streamfunction formulation of the time-dependent Navier--Stokes
%  equations, coupled to two advection equations for temperature and salinity:
% \begin{align*}
%   \frac{\partial \zeta}{\partial t} + \nabla^{\perp} \psi \cdot \nabla \zeta &= \frac{\textrm{Ra}}{\textrm{Pr}}\left(\frac{\partial T}{\partial x} - \frac{1}{R_{\rho}^0} \frac{\partial S}{\partial x}\right) + \nabla^2 \zeta, \\
%   \frac{\partial T}{\partial t} + \nabla^{\perp} \psi \cdot \nabla T &= \frac{1}{\textrm{Pr}} \nabla^2 T, \\
%   \frac{\partial S}{\partial t} + \nabla^{\perp} \psi \cdot \nabla S &= \frac{1}{\textrm{Sc}} \nabla^2 S, \\
%   \nabla^2 \psi &= \zeta.
% \end{align*}
%  $\zeta$ is the vorticity, $\psi$ is the streamfunction,
%  $T$ is the temperature, $S$ is the salinity, and $\textrm{Ra}$,
%  $\textrm{Sc}$, $\textrm{Pr}$ and ${R_{\rho}^0}$ are parameters.
%  The configuration consists of two well-mixed layers (i.e., of homogeneous
%  temperature and salinity) separated by an interface. The instability is
%  activated by a sinusoidal perturbation to the initial
%  salinity field.
% 
%  \vspace{1.0cm}
%     \begin{tcolorbox}
% \begin{minted}{python}
% from dolfin import *
% from dolfin_adjoint import *
% # ...
% gst = compute_gst("InitialSalinity", "FinalSalinity", nsv=2)
% \end{minted}
%     \end{tcolorbox}
% \end{minipage}
% \hspace{0.5cm}
% \begin{minipage}{0.38\textwidth}
% \includegraphics[width=0.48\textwidth]{images/salt-fingering}
% \includegraphics[width=0.48\textwidth]{images/salt-fingering-gst-initial}
% \begin{minipage}{0.48\textwidth}
%     \centering
% Initial salinity
% \end{minipage}
% \begin{minipage}{0.48\textwidth}
%     \centering
%     \vspace{0.3cm}
% Leading initial salinity perturbation
%     \vspace{0.3cm}
% \end{minipage}
% \begin{minipage}{0.5\textwidth}
% \begin{minipage}{0.9\textwidth}
%     \vspace{3cm}
% \footnotesize
% \textbf{$\triangleleft$ Code}: Implementation excerpt (the full code uses
% SLEPc and has $144$ lines)
% \end{minipage}
% \end{minipage}
% \begin{minipage}[t]{0.48\textwidth}
% \includegraphics[width=\textwidth]{images/salt-fingering-gst-final}
%     \centering
% Leading final salinity perturbation
% \end{minipage}
% \end{minipage}
% \end{minipage}
% };
% 
% \node[anchor=west, rounded corners=3mm, fill=simula]
% (stab title)
% at ($ (stab.north west)!3cm!(stab.north east) $)
% {\ffmfamily Generalized stability analysis};
% 
% %%%% performance %%%%
% \node[draw=simula, line width=1mm, anchor=north west,
%       rounded corners=4mm, inner sep=1cm,
%       minimum height=1cm]
% (performance)
% at ($(howitworks.south west)+(0,-2cm)$)
% {\begin{minipage}{51cm}
% \footnotesize
% \vspace{1cm}
% dolfin-adjoint runs naturally in parallel, and inherits the
% scalability and code optimizations of FEniCS.
% To verify this, we benchmarked the sensitivity analysis and generalized stability
% application examples.
% \vspace{1cm}
% 
% \begin{minipage}{0.5\textwidth}
% \begin{table}
%     \caption*{Sensitivity analysis example}
% \scriptsize
% \centering
% \begin{tabular}{@{}Tccccccccc}
% \toprule
% CPUs & 1 & 2 & 4  & Optimal\\
% Forward runtime (s) & 40.3 & 19.6 & 13.2 \\
% Adjoint runtime (s) & 39.1 & 19.3 & 12.5 \\
% %Gradient runtime (s) & 43.1 & 22.0 & 14.3 \\
% Adjoint/Forward ratio & 0.97 & 0.99 & 0.95 & \textcolor{red}{1.00} \\
% \bottomrule
% \end{tabular}
% \end{table}
% \end{minipage}
% \begin{minipage}{0.5\textwidth}
% \begin{table}
%     \caption*{Generalized stability example}
% \scriptsize
% \centering
% \begin{tabular}{@{}Tccccccccc}
% \toprule
% CPUs & 1 & 2  & Optimal\\
% Forward runtime (s) & 92.4 & 55.0 \\
% Adjoint runtime (s) & 41.4 & 25.9 \\
% Adjoint/Forward ratio & 0.45 & 0.47 & \textcolor{red}{0.5}\\
% \bottomrule
% \end{tabular}
% \end{table}
% \end{minipage}
% \vspace{0.5cm}
% 
% \textbf{Tables:} {The sensitivity analysis example is linear, while
% the generalized stability analysis example is nonlinear and converges on average
% in 2 Newton-iteration per timestep. Hence the adjoint model is expected to be
% twice as fast as the forward model.}
% 
% \end{minipage}};
% 
% \node[anchor=east, rounded corners=3mm, fill=simula]
% (performance title)
% at ($ (performance.north east)!3cm!(performance.north west) $)
% {\ffmfamily Performance};
% dolfin-adjoint supports
% 
% 
% %%%% Checkpointing %%%%
% \node[draw=simula, line width=1mm, anchor=north west,
%       rounded corners=4mm, inner sep=1cm,
%       minimum height=18cm]
% (checkpointing)
% at ($(performance.south west)+(0,-2cm)$)
% {\begin{minipage}{51cm}
% \begin{minipage}{0.57\textwidth}
% \footnotesize
% The adjoint equations depend on the forward solutions. However,
% storing the entire forward trajectory is infeasible for
% large, time-dependent simulations. In this case, dolfin-adjoint can
% employ a binomial checkpointing
% strategy via the revolve library. When activated, dolfin-adjoint automatically saves
% state checkpoints and uses them to recompute missing forward states to trade
% off memory requirements and computational effort.
% This allows for solving adjoint equations even for large-scale simulations.
% For instance, $390$ checkpoints allow
% simulations with $10^7$ time-steps at a cost of a $3 \times$ slow-down.
% 
% %\begin{table}
% %\scriptsize
% %\centering
% %\begin{tabular}{@{}Tccccccccc}
% %\toprule
% %Number of timesteps & $10^2$ & $10^3$ & $10^4$ & $10^5$ & $10^5$ & $10^7$ \\
% %Required checkpoints & 7 & 17 & 38 & 83 & 180 & 390 \\
% %\bottomrule
% %\end{tabular}
% %\end{table}
% %Checkpointing allows us to compute adjoint
% %solutions even for large-scale simulations.
% %
% \end{minipage}
% \hfill
% \begin{minipage}{0.4\textwidth}
% \begin{center}
%     \begin{figure}
%         \centering
% \vspace{0.4cm}
% \includegraphics[height=10cm]{images/revolve}
% \caption*{\textbf{Figure:} Visualisation of the optimal checkpointing strategy with 10 time levels
% and 3 checkpoints}
%     \end{figure}
% \end{center}
% \end{minipage}
% 
% \footnotesize
% \vspace{0.8cm}
% To benchmark the checkpoint implementation, we used the sensitivity example
% to compare the additional computational cost of checkpointing with a
% store-all strategy in dolfin-adjoint:
% 
% \begin{table}
%     \caption*{Slow-down factor with 11 timesteps and varying memory
%     checkpoints}
% \scriptsize
% \centering
% \begin{tabular}{@{}Tcccccccc}
% \toprule
% Theoretical adjoint to forward runtime ratio & $5.00$ & $2.18$ &
% $1.63$ & $1.45$ & $1.00$ \\
% Observed adjoint to forward runtime ratio & 5.07 & 2.26 & 1.73 & 1.53 &
% 0.90 \\
% \bottomrule
% \end{tabular}
% \end{table}
% 
% \end{minipage}};
% 
% \node[anchor=east, rounded corners=3mm, fill=simula]
% (checkpointing title)
% at ($ (checkpointing.north east)!3cm!(checkpointing.north west) $)
% {\ffmfamily Checkpointing};
% 
% %%%% how to get started %%%%
% \node[anchor=north west]
% (howtogetstarted title)
% at ($(checkpointing.south west)-(1cm,1cm)$)
% {\bfseries\ffmfamily\Large\underline{How to get started}};
% 
% \node[draw=none, anchor=north west]
% (howtogetstarted)
% at ( $(howtogetstarted title.south west)+(1cm,0)$ )
% {
% \vspace{1cm}
% \begin{minipage}{20cm}
%     %\includegraphics[height=3cm]{logodolfinadjoint}
%         \normalsize
%         http://dolfin-adjoint.org
% \end{minipage}
% \begin{minipage}{30cm}
%         \footnotesize
%         Contains an introduction to adjoints, documentation, tutorials and
%         installation instructions for Linux (with Ubuntu packages) and MacOS X.
% \end{minipage}
% };
% 
% %%%% references %%%%
% %\node[anchor=north west]
% %(reference title)
% %at ($(stab.south west)-(0cm,2cm)$)
% %{\bfseries\ffmfamily\Large\underline{References}};
% %
% %\node[draw=none, anchor=north west]
% %(reference text)
% %at ( $(reference title.south west)+(1cm,0)$ )
% %{\begin{minipage}{37cm}
% %\footnotesize
% %S.~Pezzuto, D.~Ambrosi, A.~Quarteroni.
% %An orthotropic active--strain model for myocardium mechanics and its numerical
% %approximation.
% %\textit{European Journal of Mechanics--A/Solids}, 2014.
% %
% %\medskip
% %G.~Holzapfel, R.~Ogden.
% %Constitutive modelling of passive myocardium: a structurally based framework
% %for material characterization.
% %\textit{Philosophical Transactions of the Royal Society A: Mathematical,
% %Physical and Engineering Sciences}, 2009.
% %\end{minipage}};
% 
% %%%% software %%%%
% \node[anchor=north west]
% (software title)
% %at ($(stab.south west)-(1cm,0cm)$)
% at ($(howtogetstarted.south west)-(1cm,1cm)$)
% {\bfseries\ffmfamily\Large\underline{Downloads}};
% 
% \node[draw=none, anchor=north west]
% (software text)
% at ( $(software title.south west)+(0cm,0)$ )
% {\begin{minipage}{60cm}
% \footnotesize
% \begin{minipage}[t]{3cm}
% \vskip0pt
% \includegraphics[height=3cm]{qrposter}
% \end{minipage}
% \begin{minipage}[t]{5cm}
% \vskip0pt
% This poster \\PDF format
% \end{minipage}
% \hspace{3cm}
% \begin{minipage}[t]{3cm}
% \vskip0pt
% \includegraphics[height=3cm]{qrreferences}
% \end{minipage}
% \begin{minipage}[t]{12cm}
% \vskip0pt
% References \\
% dolfin-adjoint.org/citing
% \end{minipage}
% \vspace{1cm}
% \begin{minipage}[t]{3cm}
% \vskip0pt
% \includegraphics[height=3cm]{qrdolfinadjointsource}
% \end{minipage}
% \begin{minipage}[t]{12cm}
% \vskip0pt
% Source code \\
% bitbucket.org/dolfin-adjoint
% \end{minipage}
% \hspace{2cm}
% \begin{minipage}[t]{3cm}
% \vskip0pt
% \includegraphics[height=3cm]{qrfenics}
% \end{minipage}
% \begin{minipage}[t]{5cm}
% \vskip0pt
% \includegraphics[height=3cm]{logofenics}
% \end{minipage}
% \hspace{3cm}
% 
% \end{minipage}};
% 
% % Acknowledgments
% %\node[anchor=west]
% %at ( $(software text.south west)+(0cm,0cm)$ )
% %{
% %    \footnotesize
% %    This work is supported by NERC and EPSRC in the UK, and the
% %    Center for Biomedical Computing in Norway.
% %    %\includegraphics[height=4cm,clip]{nerc-logo}
% %    %\includegraphics[height=4cm,clip]{epsrc}
% %};
% \node[anchor=east]
% at ( $(stab.south east)+(0cm,-1.5cm)$ )
% {
%     \includegraphics[height=2.2cm,clip]{CBClogoII}
%     \hspace{0.3cm}
%     \includegraphics[height=2.2cm,clip]{epsrc}
%     \hspace{0.3cm}
%     \includegraphics[height=2.2cm,clip]{nerc-logo}
% };

%%%% footer %%%%
\begingroup
\draw[fill=eyellow, draw=none] { 
  ($(current bounding box.south west)+(0cm, 43.5mm)$) --
  ($(current bounding box.south east)+(0cm, 43.5mm)$) -- 
  (current bounding box.south east) --
  (current bounding box.south west) -- 
  cycle
};
   %
\draw[elblue, line width=3.5mm]{
  ($(current bounding box.south east)-(0cm, -1.75mm)$) --
  ($(current bounding box.south west)-(0cm, -1.75mm)$)
};
   %
 \draw[elblue, line width=3.5mm]{
   ($(current bounding box.south east)-(0cm, -41.75mm)$) --
   ($(current bounding box.south west)-(0cm, -41.75mm)$)
 };
 \node[anchor=west]
 at ( $(current bounding box.south west)+(1cm, 21.75mm)$ )
 {\includegraphics[height=4cm,trim=0 0 370 0,clip]{graphics/uiologo}};
    %
\node[anchor=east]
at ( $(current bounding box.south east)+(-1cm, 26.75mm)$ )
{
    \includegraphics[height=3.0cm,trim=0 0 0 0, clip]{graphics/cbclogo}
};
\endgroup

\end{tikzpicture}

\end{document}
